\documentclass[10pt]{article}
\usepackage[utf8]{inputenc}
\usepackage{amsmath}
\usepackage{amssymb}
\usepackage{graphicx}
\usepackage{geometry}
\usepackage{enumitem}
\usepackage{multicol}
\usepackage{tabularx}
\usepackage[table,xcdraw]{xcolor}
\usepackage{array}

\usepackage[table]{xcolor}
\usepackage{tabularx}
\usepackage{booktabs}
\usepackage{amsmath}

\usepackage{ifthen}
\newboolean{showresults}
\setboolean{showresults}{true}

\geometry{a4paper, margin=1in}

\pagestyle{empty}

\title{\vspace{-1.5cm}\Large\textbf{University of Nebraska-Lincoln}\\[0.2cm]
\Large\textbf{Digital Signal Processing: Assignment 1}} 
\date{\vspace{-0.5cm}\small\textbf{Due Date: September 19, 2025}}

\begin{document}

\maketitle


% \noindent 
% For the following questions, assume a discrete time system denoted by $H$ that provides an input/output mapping for an input sequence $x[n]$ and an output sequence $y[n]$.
% \\

\noindent
For each of the following input/output relations for a discrete time system, please indicate whether the system is:


\begin{itemize}[left=0pt, labelwidth=*, align=left]
    \item[\bf{L} or \bf{NL}:] Linear or Not Linear
    \item[\bf{TI} or \bf{NTI}:] Time Invariant or Not Time Invariant
    \item[\bf{C} or \bf{NC}:] Causal or Not Causal
    \item[\bf{B} or \bf{NB}:] BIBO Stable or Not BIBO Stable
\end{itemize}

\noindent 
If for any property it is not possible to say, then indicate this by writing \textbf{CBD} (Cannot Be Determined).
\vspace{0.5cm}


\ifthenelse{\not{\boolean{showresults}}}{
\renewcommand{\arraystretch}{1.5} % Adjusts row height
\begin{table}[h!]
\centering
\begin{tabularx}{\textwidth}{|>{\raggedright\arraybackslash}X|c|c|c|c|}
    \hline
    \textbf{Input/Output Relation} & \textbf{L/NL} & \textbf{TI/NTI} & \textbf{C/NC} & \textbf{B/NB} \\
    \hline
    $y[n] = 7x[n+2]$ & & & & \\
    \hline
    $y[n] = x[n] - 4x[n-5] + 3x[n-15]$ & & & & \\
    \hline
    $y[n] = \dfrac{x[n-4]}{x[n]+8}$ & & & & \\
    \hline
    $y[n] = x[n] + (x[n-3])^{-1} + \cos(n+4)$ & & & & \\
    \hline
    $y[n] = \operatorname{med}\{x[n+2], x[n], x[n-2]\}$ & & & & \\
    \hline
    $y[n] = x[1] + \sum_{k=-\infty}^n x[k]$ & & & & \\
    \hline
    $y[n] = \dfrac{1}{5} \sum_{k=n-4}^{n+6} x[k]$ & & & & \\
    \hline
    $y[n] = \sum_{k=0}^{\infty} h[k]x[n-2k]$ & & & & \\
    \hline
    $y[n] = \sum_{k=2}^{n+4} x[k]\left(\dfrac{3}{5}\right)^k$ & & & & \\
    \hline
    $y[n] = \sum_{k=3}^{n} x[k]\left(\dfrac{2}{k}\right)(-1)^k$ & & & & \\
    \hline
\end{tabularx}
\caption{Table of Input/Output Relations}
\end{table}
}{}



\ifthenelse{\boolean{showresults}}{

\renewcommand{\arraystretch}{1.5} 
\begin{table}[h!]
\centering
\begin{tabularx}{\textwidth}{|>{\raggedright\arraybackslash}X|c|c|c|c|}
    \hline
    \textbf{Input/Output Relation} & \textbf{L/NL} & \textbf{TI/NTI} & \textbf{C/NC} & \textbf{B/NB} \\
    \hline
    $y[n] = 7x[n+2]$ 
    & L    % Linear
    & TI   % Time-Invariant
    & NC   % Noncausal (depends on future input)
    & B   % Not Bounded (if input unbounded, output unbounded)
    \\
    \hline
    $y[n] = x[n] - 4x[n-5] + 3x[n-15]$ 
    & L 
    & TI 
    & C 
    & B 
    \\
    \hline
    $y[n] = \dfrac{x[n-4]}{x[n]+8}$ 
    & NL   % Nonlinear (division by input)
    & TI 
    & C 
    & NB 
    \\
    \hline
    $y[n] = x[n] + (x[n-3])^{-1} + \cos(n+4)$ 
    & NL   % Nonlinear (inverse term)
    & NTI  % Not time-invariant (cos(n+4) is time-varying)
    & C 
    & NB 
    \\
    \hline
    $y[n] = \operatorname{med}\{x[n+2], x[n], x[n-2]\}$ 
    & NL   % Median is nonlinear
    & TI 
    & NC   % Depends on future input (x[n+2])
    & B    % Bounded if input is bounded
    \\
    \hline
    $y[n] = x[1] + \sum_{k=-\infty}^n x[k]$ 
    & L 
    & NTI 
    & NC   % Depends on x[1] regardless of n (noncausal for n < 1)
    & NB   % The sum can diverge
    \\
    \hline
    $y[n] = \dfrac{1}{5} \sum_{k=n-4}^{n+6} x[k]$ 
    & L 
    & TI 
    & NC   % Needs future input up to n+6
    & B    % Bounded if input is bounded
    \\
    \hline
    $y[n] = \sum_{k=0}^{\infty} h[k]x[n-2k]$ 
    & L 
    & TI 
    & C    % Only depends on present and past (if h[k]=0 for large k, otherwise could be debated)
    & NB   % If h[k] or x[n] is unbounded, sum can diverge
    \\
    \hline
    $y[n] = \sum_{k=2}^{n+4} x[k]\left(\dfrac{3}{5}\right)^k$ 
    & L 
    & NTI  % Upper limit depends on n
    & NC   % Depends on future input up to n+4
    & B   % Not necessarily bounded if x[k] is not bounded
    \\
    \hline
    $y[n] = \sum_{k=3}^{n} x[k]\left(\dfrac{2}{k}\right)(-1)^k$ 
    & L 
    & NTI   % Limits only depend on n, but summand does not depend on n directly
    & C    % Only depends on present and past
    & NB   % Again, not necessarily bounded if x[k] is not bounded
    \\
    \hline
\end{tabularx}
\caption{Classification of Input/Output Relations with Properties}
\end{table}



}{}




\noindent \textbf{Notes:}
\begin{enumerate}[left=0pt, labelwidth=*, align=left]
    \item In our notation, if the upper limit of a summation is higher than or equal to the lower limit, a summation occurs; otherwise, the summation returns a zero.
    \item Assume $|h[n]|$ is bounded for all $n$.
    \item $\operatorname{med}\{\}$ returns the median of three values -- e.g., $\operatorname{med}\{7, -3, 10\} = 7$.
    \item Attach any work you did to obtain your answers. You can either show that the relation meets our definition, or show a counter-example that shows the relation does not. 
\end{enumerate}

\vfill

\end{document}